% 本文件是示例论文的一部分
% 论文的主文件位于上级目录的 `main.tex`

\chapter{快速上手}

\section{欢迎}

欢迎使用 \nwafuthesis,本文档将介绍如何利用 \nwafuthesis 模板进行学位论文写作,
假设读者有 \LaTeX 写作经验,并会使用搜索引擎解决常见问题。

源代码托管于 \url{https://github.com/registor/nwafuthesis},
欢迎来提 issue/PR。

\section{\LaTeX 环境准备}

由于本模板使用了大量宏包,因此对 \LaTeX 环境有不少要求。
推荐使用以下打 \ding{51} 的 \LaTeX 发行版:
\begin{itemize}
\item[\ding{51}]\TeX~Live 请安装以下 collection:langchinese, latexextra, science, pictures, fontsextra;\\
如果觉得安装体积太大的话,可以看 \texttt{.ci/texlive.pkgs} 列出的所需宏包;
\item[\ding{51}]MiK\TeX 可能国内镜像服务器无法联通,如果无法联通,建议隔天再试; \\
因为 MiK\TeX{} 能自动下载安装宏包,推荐 Windows 用户使用。
\item[\ding{53}]CTeX  \qtbr{不推荐},可能会有宏包缺失、版本过旧导致无法编译现象。
\end{itemize}

\note{本模板基于\TeX~Live 2018开发完成,强烈建议安装最新版
  \TeX~Live 发行版。}

\section{编译模板和文档}

只有在找不到 \verb|nwafuthesis.cls| 文件的时候,才需要执行本步骤。

进入模板的根目录,运行 \verb|build.bat|(Windows) 或 \verb|build.sh|(其他系统),
它会生成模板 \verb|nwafuthesis.cls| 以及对应的说明文档 \verb|nwafuthesis.pdf|。

\section{使用模板}

论文写作时,请确认\textbf{论文的目录}(\verb|main.tex|所在的目录)下有以下文件:
\begin{itemize}
  \item \verb|nwafuthesis.cls| 文档模板;
  \item \verb|logo/| 文件夹,内含学校的LOGO图标;
\end{itemize}

如果论文目录下没有这些文件的话,请从本模板根目录复制一份。

\section{开始写作}

最方便的开始方法,莫过于修改现有的示例文稿,因此强烈推荐直接demo下的文
档实现学位论文撰写。

在撰写学位论文时,强烈建议按\autoref{fig:dirforest}所示的目录结构组织和管理写作过程中的各个文件:

\begin{figure}[htb]
  \begin{forest}
    pic dir tree,
    pic root,
    for tree={% folder icons by default; override using file for file icons
      directory,
    },
    [jobname【工作根目录】%, 
      [bib【参考文献数据库目录,根据需要,可以有多个数据库】%
        [sample.bib【样例数据库,根据需要,可以有多个数据库】, file%
        ]   
      ]
      [contents【各章节内容的\LaTeX 源文件,可根据需要增减】        
        [ack.tex【致谢】, file
        ]
        [chap00-abs【摘要】.tex, file
        ]
        [chap01.tex【第1章】, file
        ]
        [$\vdots$, file
        ]
        [denotation.tex【主要符号对照表】, file
        ]
        [resume.tex【个人简历】, file
        ]
      ]
      [data【数据文件,可根据需要增减】        
        [ackdata.csv【资助项目信息数据文件】, file
        ]
        [committeememb.csv【答辩委员信息数据文件】, file
        ]
      ]  
      [figs【插图目录,可根据需要增减】
        [plot
        ]
        [xxxx.png, file
        ]
        [xxxx.pdf, file
        ]
      ]
      [logo【学校校徽图标,\emph{必须存在},且置于根目录】        
        % [cie.png, file
        % ]
        % [nwafu-bar.png, file
        % ]
        [nwafubilogo.png, file
        ]
        [nwafu-circle.png, file
        ]
        % [NWAFU\_logo.png, file
        % ]
        % [nwsuaf\_logo\_new.png, file
        % ]
      ]
      [setup【自定义命令、环境、引入宏包等\LaTeX 源文件,可根据需要调整】        
        [format.tex【自定义命令、环境、参数设置等】, file
        ]
        [packages.tex【引入宏包】, file
        ]
      ]
      [gb7714-2015ay.bbx【参考文献著录样式文件,\emph{必须存在},且置于根目录】, file
      ]
      [gb7714-2015ay.cbx【参考文引注样式文件,\emph{必须存在},且置于根目录】, file
      ]
      [main.tex【主控文件,\emph{必须存在},且置于根目录】, file
      ]
      [Makefile【make命令需要的文件,若不执行make命令,可以不需要】, file
      ]
      [nwafuthesis.cls【文档类文件(模板文件),\emph{必须存在},且置于根目录】, file
      ]
      [zhlineskip.sty【中文行距宏包,若\TeX~Live 2018未及时更新,必须存在,且置于根目录】, file
      ]
      [exboxie.sty【\LaTeX 示例代码宏包,如不需要,可以删除,请注意不能再使用相应命令】, file
      ]       
    ]
  \end{forest}
  \caption{学位论文撰写目录结构}
  \label{fig:dirforest}
\end{figure}


% \begin{itemize}
%   \item \verb|main.tex| 主文件,定义了文档包含的内容,建议不要随意更改
%     \verb|main.tex|文件名,其它的\verb|*.tex|中会用到该文件名信息;
%   \item \verb|setup/packages.tex| 载入需要的宏包,可根据需要进行增
%     加或删除;
%   \item \verb|setup/format.tex| 全局使用的自定义命令、相关设置等;  
%   \item \verb|content/*.tex| 文件夹,按章节拆分的文档内容,分章节以存
%     放在这里;
%   \item \verb|figs/| 文件夹,插图文件;  
%   \item \verb|bib/| 文件夹,内含参考文献数据库,文献数据库是纯文本文
%     件,请务必采用\qtbr{UTF8编码}存储;
%   \item \verb|ref/| 文件夹,可有可无,内含写作过程中用到的资料、参考文
%     献、记录等;  
% \end{itemize}

完成部分或所有\verb|*.tex|撰写和修改后,可以在命令行使用 \verb|latexmk -xelatex main|
进行编译输出\verb|main.pdf|文件,可以根据需要对结果\texttt{pdf}文件进行改名。

也可以使用\texttt{TeXstudio}、\texttt{vscode}等图形界面的编辑器的进行
编译输出。

\section{打印论文}

如果论文需要双面打印的话,请务必修改文档类选项,编译双面打印用的 PDF 文件。
具体地说,在主文件的头部,去除 \texttt{openany, oneside},改成 \texttt{openright, blankleft, twoside}。

%%% Local Variables: 
%%% mode: latex
%%% TeX-master: "../main.tex"
%%% End:
