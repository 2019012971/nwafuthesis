\chapter{后记}

\section{吐槽}

\verb!\begin{轻松+愉快}!

做模板过程中遇到的大问题,在于如何正确理解学校对论文格式的要求。
虽然有《本科毕业设计(论文)撰写格式要求》、《研究生学位论文撰写要求》,
但这些要求依然不够细致,因为那些要求都是假定你用 Word 来写论文的,要求里的内容是 Word 设置的操作方法,
所以还要先学习 Word 的排版算法,因此,本模板
但还有很多细节部分,因为能力有限,没能实现。

最后容我吐槽一下学校的 Word 模板,那个 Word 模板可能从最初做出来后,就基本没有变化。
很多编号的事情都要由手工来完成,比如说目录页码、
各级标题的编号、题注等。这些完全可以自动编号的工作,如果要手工做的话是非常累人和容易出错的。

同时,强烈建议学校能用标准的地方一定要用标准,比如参考文献的GB7714-2015标准!

\section{明天}

转眼间n年过去,又到了写毕业论文的时候了,一直想完成我们学校的毕业论文模板,今天总算有了一个初稿。

目前, \nwafuthesis{} 应该还有相当多的问题,但没有用户的话,由于作者能力有限,很难发现这些问题,
还请各位使用 \nwafuthesis{} 的先行者们(Pioneers) 能及时反馈意见和建议。

愿所有使用 \nwafuthesis{} 的人,不会被评审老师指责格式问题。
