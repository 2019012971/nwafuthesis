\iffalse
  % 本块代码被上方的 iffalse 注释掉,如需使用,请改为 iftrue
  % 使用 Noto 字体替换中文宋体、黑体
  \setCJKfamilyfont{\CJKrmdefault}[BoldFont=Noto Serif CJK SC Bold]{Noto Serif CJK SC}
  \renewcommand\songti{\CJKfamily{\CJKrmdefault}}
  \setCJKfamilyfont{\CJKsfdefault}[BoldFont=Noto Sans CJK SC Bold]{Noto Sans CJK SC Medium}
  \renewcommand\heiti{\CJKfamily{\CJKsfdefault}}
\fi

% ==============LaTeX命令排版命令========================
\newcommand\cs[1]{\texttt{\textbackslash#1}}
%\newcommand\pkg[1]{\texttt{#1}\textsuperscript{PKG}}
%\newcommand\env[1]{\texttt{#1}}
\newcommand{\note}[1]{{%
  \color{magenta}{\bfseries 注意:}\emph{#1}}}

% ==============自定义定理环境========================
\theoremstyle{nwafuplain}
\nwafutheoremchapu{definition}{定义}
\nwafutheoremchapu{assumption}{假设}
\nwafutheoremchap{exercise}{练习}
\nwafutheoremchap{nonsense}{胡诌}
\nwafutheoremg[句]{lines}{句子}

% 设置插图目录
\graphicspath{{./figs/},{./logo/}}

% 重定义强调字体的代码
% 在此设置为加粗,注意需要使用etoolbox宏包
\makeatletter
\let\origemph\emph
\newcommand*\emphfont{\normalfont\bfseries}
\DeclareTextFontCommand\@textemph{\emphfont}
\newcommand\textem[1]{%
  \ifdefstrequal{\f@series}{\bfdefault}
    {\@textemph{\CTEXunderline{#1}}}
    {\@textemph{#1}}%
}
\RenewDocumentCommand\emph{s o m}{%
  \IfBooleanTF{#1}
    {\textem{#3}}
    {\IfNoValueTF{#2}
      {\textem{#3}\index{#3}}
      {\textem{#3}\index{#2}}%
     }%
}
\makeatother   

%% 自定义相关的名称宏命令
%% ==================================================
%% \newcommand{\yourcommand}[参数个数]{内容}
% 西北农林科技大学各单位名称
\newcommand{\nwsuaf}{西北农林科技大学}
\newcommand{\cie}{信息工程学院}
\newcommand{\ca}{农学院}
\newcommand{\cpp}{植物保护学院}
\newcommand{\ch}{园艺学院}
\newcommand{\cast}{动物科技学院}
\newcommand{\cvm}{动物医学院}
\newcommand{\cf}{林学院}
\newcommand{\claa}{风景园林艺术学院}
\newcommand{\cnre}{资源环境学院}
\newcommand{\cwrae}{水利与建筑工程学院}
\newcommand{\cmee}{机械与电子工程学院}
\newcommand{\cfse}{食品科学与工程学院}
\newcommand{\ce}{葡萄酒学院}
%\newcommand{\cls}{生命科学学院}
\newcommand{\cst}{理学院}
\newcommand{\ccp}{化学与药学院}
\newcommand{\cem}{经济管理学院}
\newcommand{\cm}{马克思主义学院}
\newcommand{\dfl}{外语系}
\newcommand{\iec}{创新实验学院}
\newcommand{\ci}{国际学院}
\newcommand{\dpe}{体育部}
\newcommand{\cvae}{成人教育}
\newcommand{\iswc}{水土保持研究所}

% ==============自定义的双引号、字体强调等命令========================
% 定义提醒字体
\newcommand{\alert}[1]{\textcolor{red}{\textbf{#1}}}
% 定义引号命令
\newcommand{\qtmark}[1]{{\symbol{"201C}}#1{\symbol{"201D}}}
%\newcommand{\qtmark}[1]{``#1''}%``''
% 定义带引号的加粗强调命令
\newcommand{\qtb}[1]{\qtmark{\emph{#1}}}
% 定义带引号的加粗加红强调命令
\newcommand{\qtbr}[1]{\qtmark{\emph{\alert{#1}}}}


%%% Local Variables: 
%%% mode: latex
%%% TeX-master: "../main.tex"
%%% End:
