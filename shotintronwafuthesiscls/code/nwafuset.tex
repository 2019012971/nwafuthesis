% 设置文档基本信息,\linebreak 前面不要有空格,否则在无需换行的场合,中文之间的空格无法消除
% 另,在\nwafuset中不可以出现空行
\nwafuset{
  clscode = {TP391.9},                     % 分类码,仅研究生需要
  udccode = {004.9},                       % UDC码,仅研究生需要
  cfdlevel = {公开},                       % 密级,仅研究生需要(只能取公开、限制、秘密、机密、绝密五个等级)
  unvcode = {10712},                       % 学校代码,仅研究生需要,西北农林科技大学取10712
  studentid = {2013051289},                % 学号,本科/研究生需要
  gradyear = {2019},                       % 毕业年,本科/研究生需要
  title = {\nwafuthesis{} 快速上手示例文档}, % 论文题目,本科/研究生需要
  professionaltype = {工程硕士},            % 专硕类型,仅专硕需要
  professionalfield = {软件工程},           % 专硕领域,仅专硕需要
  majorsubject = {计算机应用技术},           % 学科专业,仅研究生需要
  researchfield = {智能媒体处理},            % 研究方向,仅研究生需要
  researchername = {\LaTeX{}er},           % 研究生论文作者姓名,仅研究生需要
  major = {计算机科学与技术},               % 专业,仅本科生需要
  advisers = {耿楠},                      % 指导教师姓名,本科/研究生需要
  coadvisers = {Donald Knuth\quad 大师},  % 协助指导教师姓名,本科/研究生需要
  classid = {152},                       % 班级号,仅本科生需要(只填写数字,不要有其它内容)
  author = {\LaTeX{}er},                 % 论文作者姓名,仅本科生需要
  college = {信息工程学院},               % 学院名称,仅本科生需要 
  applydate = {\today},                  % 完成日期(默认为当前日期),本科/研究生需要
  defensedate = {\today},                % 答辩日期(默认为当前日期),研究生需要
  adviserteam = {耿楠,Knuth,Lamport},    % 指导小组,博士论文需要(不同人名用英文逗号分割)  
  cmteemembfile = {data/committeememb.csv}, % 答辩委员会成员csv文件名称,包含相对路径,研究生论文需要  
  ackdatafile = {data/ackdata.csv},         % 资助项目csv文件名称,包含相对路径,研究生论文需要
}
