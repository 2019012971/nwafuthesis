% 加载宏包
%===================注意======================%
% 在调用beamer.cls宏包后,以下宏包将自动调用,
% 不应单独调用这些宏包,以免发生冲突
% amsfonts, amsmath, amssymb, amsthm, 
% enumerate, geometry, graphics, graphicx, 
% hyperref, url, 
% ifpdf, keyval, xcolor, xxcolor
% =============================================%
% 由于setspace宏包会改变\@footnotetext,从面造成footcite不能引用的问题,
% 以下代码用于修正这一问题
\usepackage{etoolbox}
% 代码排版工具宏包
% 需要预先安装 python 和 pygments。
% 此宏包要加  -shell-escape 编译参数
% 版本2.0,支持行内代码排版
\usepackage{minted}

%\usepackage[bwr]{callouts}

% ++++++++++++++++++++++++++++++++++++++++++++++++++
% 为一部分代码用于解决引用setspace调整间距宏包后造成的参考文献引用脚注
% 丢失问题
\makeatletter
% save the meaning of \@footnotetext
\let\BEAMER@footnotetext\@footnotetext
\makeatother
% 调整间距
\usepackage{setspace} 
% 叉号与对号要用到的字体
%\usepackage{pifont}
% 绘图
\usepackage{tikz}
\usepackage{pgfplots}
% 部分latex的Logo
\usepackage{xltxtra}

\usepackage{fontawesome5}

\usepackage{multicol}

%% 设置绘制目录结构的宏及参数
\usepackage[edges]{forest}

% ========排版键盘组合和菜单的宏包=========
\usepackage{menukeys}

%%% Local Variables: 
%%% mode: latex
%%% TeX-master: "../main.tex"
%%% End:

