
    \section{测试:专著/专著中的析出文献/标准}
\begin{refsection}

\begin{enumerate}
  \item 中文文献存在信息缺省的情况测试
\cite{booknodate,booknolocation,booknopages,booknopublisher,booknopublishernoyear,booknopublisherpage, 余敏2001-179-193,余敏2001-179-193a,余敏2001-179-193b,余敏2001-179-193c,booknoauthor}
  \item 英文文献信息缺省情况以及英文个人作者判断测试
\cite{Parsons2000a--,Parsons2000b--,Parsons2000--,Parsons2000nodate--,
Parsons2000noloc--,Parsons2000nopub--,Parsons2000nopubnoyear--,
Parsons2000nopubpages--,Parsons2000noauthor--}
  \item 年份信息有额外说明的文献比较测试\cite{汪昂1912--,汪昂1881--,王夫之1845--}

  \item 专著带前后缀的作者名\cite{Peebles2001-100-100}

  \item 专著更多文献测试
\cite{GPS1988--}\cite{顾炎武1982--}\cite{PIGGOT1990--}\cite{PEEBLES2001--}
\cite{Poisel2013--}\cite{张伯伟2002--}\cite{2009-155-155}
\cite{GPS1988--,CRANE1972--,CRAWFPRD1995--,Gradshteyn2000--,IFLAI1977--,Kinchy2012-50-50,Lin2004--,Miroslav2004--,Mittelbach2004--,OBRIEN1994--}
\cite{PEEBLES2001--,Peebles2001-100-100,PIGGOT1990--,Poisel2013--,Praetzellis2011-13-13,Proakis2007--,Reed2005--,ROOD2001--,Ross2010--,Simon2004--,Simon2001--,Stueber2001--,Yi2013--,YUFIN2000--, 埃伯哈德$ullet$蔡德勒等2012--,蔡敏2006--,陈希孺2009--,陈志杰2006--,樊昌信2005--,顾炎武1982--,广西壮族自治区林业厅1993--,郭文彬2006--,候文顺2010-119-119,胡承正2010-112-112,胡广书2012--,胡伟2011--,蒋有绪1998--,库恩2012--,李云霞2009--,廖平2012--,刘海洋2013--,罗斯基2009--,美国妇产科医师学会2010-38-39,鸟哥--,孙玉文2000--,唐绪军1999-117-121,同济大学数学系2014--,王雪松2010--,辛希孟1994--,阎毅2013--,杨林2015--}
\cite{张伯伟2002--,赵凯华1995--,赵学功2001--,赵耀东1998--,中国企业投资协会2013--,中国社会科学院语言研究所词典编辑室1996--,庄钊文2007--,1962-50-50,2009-155-155,booknoauthor,Parsons2000noauthor--}
\end{enumerate}
\printbibliography[heading=bibliography,title=【专著】]
\end{refsection}


\begin{refsection}
\begin{enumerate}
  \item 专著的析出文献\cite{马克思2013-302-302}\cite{王夫之2011-1109-1109}
  \cite{BUSECK1980-117-211,MARTIN1996-85-96,Weinstein1974-745-772,白书农1998-146-163,陈晋镳1980-56-114,程根伟1999-32-36,楼梦麟2011-11-12,马克思1982-505-505,马克思2013-302-302,钟文发1996-468-471,1977-49-49,1988-590-590,王夫之2011-1109-1109}
\end{enumerate}
\printbibliography[heading=bibliography,title=【专著中的析出文献】]
\end{refsection}


\begin{refsection}

\begin{enumerate}
  \item 标准引用\cite{国家标准局信息分类编码研究所1988-59-92,国家环境保护局科技标准司1996-2-3, 全国广播电视标准化技术委员会2007-1-1,全国文献工作标准化委员会第七分委员会1986--,全国信息文献标准化技术委员会2010-3-3, 中华人民共和国国家质量监督检验检疫总局2015,standardinfoiso158}
\end{enumerate}
\printbibliography[heading=bibliography,title=【标准】]
\end{refsection}


\section{测试:汇编文集}
\begin{refsection}
汇编文集类似于book和inbook\cite{韩吉人1985-90-99}\cite{中国职工教育研究会1985--}

{
%\hyphenation{kurose-gawa}
%\hyphenpenalty=1000
%\tolerance=500
\printbibliography%[heading=subbibintoc,title=【参考文献】]
}

\end{refsection}

\section{测试:连续出版物及其析出文献}
\begin{refsection}

\begin{enumerate}
  \item 期刊完整引用\cite{中国地质学会1936--,中国图书馆学会1957--,AAAS1883--,中华医学会湖北分会1984--}
\end{enumerate}

\printbibliography[heading=bibliography,title=【连续出版物】]
\end{refsection}

\begin{refsection}

\begin{enumerate}
  \item 期刊文章引用和引用标签测试
  \cite{Chiani1998-2998-3008,Chiani2004-1312-1318,Chiani2004-1312-1318a,
Chiani2004-1312-1318b,Chiani2003-840-845,Chiani2009-231-254}

  \item doi和卷期样式\cite{储大同2010-721-724}

  \item 合期期刊测试\cite{储大同2010-721-724m}

  \item 报纸引用测试\cite{丁文祥2000--,傅刚2000--,刘裕国2013-01-12--,张田勤2000--}

  \item 更多测试
  \cite{Andersen1995-42-49,Andrisano1998-1383-1401,Caplan1993-61-66,Chiani1998-2998-3008,Chiani2004-1312-1318,Chiani2009-231-254,CHRISTINE1998-331-332,Coulson2004-2277-2287,Coulson2006-2484-2492,Dardari2004-1557-1567,Dardari1999-1709-1721,DESMARAIS1992-605-609,Franz2013-1053-1062,Giorgetti2005-384-389,Giorgetti2005-2139-2149,Giorgetti2005-1037-1042,Haemaelaeinen2002-1712-1721,HEWITT1984-205-218,Holtzman1992-243-247,Hu2006-1720-1724,KANAMORI1998-2063-2064,KENNEDY1975-311-386,KENNEDY1975-339-360,McEliece1984-44-53,Milstein1982-436-446,Moeneclaey2001-497-505,Molisch2006-3151-3166,Nasri2007-4090-4100,articlemorenames,Park2010-696-715,Pinto2009-1268-1282,Quek2007-2126-2139,Saito2006-169-176,Shi2007-1118-1128,Snow2007-1736-1746,STIEG1981-549-560,Walls2013-399-418,Zhang2007-500-503,Zhao2002-1684-1691, 陈高峰2011-230-232,陈建军2010-93-93,陈金成2001-1861-1864,储大同2010-721-724,储大同2010-721-724m,高光明1998-60-65, 高翔2015-26-31,江向东1999-4-4,李炳穆2000-5-8,李晓东1999-101-106,梁振兴1999-24-32,刘彻东1998-38-39,刘晨2007-400-404, 刘武1999-2481-2488,卢秋红2009-247-251,鲁明羽1998-290-295,莫少强1999-1-6,谭跃进2011-441-445,陶仁骥1984-527-527,王雪峥2013-249-254, 伍江华2010-70-74,亚洲地质图编目组1978-194-208,杨洪升2013-56-75,杨友烈1999-60-65, 于潇2012-1518-1523,詹广平2013-8-10,张敏莉2007-500-503,张庆杰2009-30-33,张晓琴2011--,周学武2013-49-52,郜宪林2001-114-116}
\end{enumerate}

\printbibliography[heading=bibliography,title=【连续出版物中的析出文献】]
\end{refsection}

\section{测试:电子资源}
\begin{refsection}
\begin{enumerate}
  \item 电子资源\cite{Commonwealth--,HOPKINSON--,OMG2003--,OCLC--,李强2012-05-03--,萧钰2001--,Alliance--,Dublin2012-06-14--,JabRef中文手册--,1989--,JabRefManual--}

  \item 仅有网址的文献
\cite{1962-50-50,2009-155-155}
\cite{olnoauthorcn}
\cite{olnoauthoren}
\cite{Allianceurlonly}

\item 注意:对于作者年制,这里有4篇文献都是noauthor,有两篇有年份可以轻易分开,还有两篇没有年份存在歧义,所以在标注中用了[n.d.]加a和b分开,但在参考文献表中,各个版本的biblatex表现是不同的,其中3.4版因为进行newbibmacro*\{date+extrayear\}的定义时候,首先判断iffieldundef\{\textbackslash thefield\{datelabelsource\}year\},当不存在datelabelsource的值+year的域时,就不再添加了。如果需要加extrayear也可以修改出来,但其实并无必要。这与标注中用的newbibmacro*\{cite:labelyear+extrayear\}(在authoryear.cbx文件中)的定义是不一样的。

\end{enumerate}

{
\hyphenation{kurose-gawa}
%\hyphenpenalty=1000
%\tolerance=500
\printbibliography%[heading=subbibintoc,title=【参考文献】]
}
\end{refsection}

\section{测试:会议论文集及其析出文献}
\begin{refsection}
会议论文集\cite{陈志勇2011--,雷光春2012--,ROSENTHALL1963--,GANZHA2000--,Babu2014--,中国力学学会1999--, 中国社会科学院台湾史研究中心2012--}

会议论文引用\cite{韩吉人1985-90-99,FOURNEY1971-17-38,FOURNEY1971-17-38a,Nemec1997-209-214, 贾东琴2011-45-52, 裴丽生1981-2-10,汪学军2002-22-25,张忠智1997-33-34}
      \cite{Choi2002-1075-1080,Dardari2002-201-206,Firoozbakhsh2003-473-477,Foerster2002-1931-1935,
      Fontana2002-309-313,Giorgetti2005-794-798,Giorgetti2006--,Li2004-21-24,Nasri2008-3616-3621,Piazzo2001--}

{
%\hyphenation{kurose-gawa}
%\hyphenpenalty=1000
%\tolerance=500
\printbibliography%[heading=subbibintoc,title=【参考文献】]
}
\end{refsection}

\section{测试:报告}
\begin{refsection}
\begin{enumerate}
  \item 技术报告引用
  \cite{Calkin2011-8-9,Eggrers--,Humphrey1971--,DTFHA1990--,WHO1970--,汤万金2013-09-30--,中华人民共和国国务院新闻办公室2013-04-16--}
  \item 手册引用
  \cite{Lehman2013--,Lehman2015,Mittelbach2015--,Oetiker2011--,Robertson2011--,Sommerfeldt2011--,Umeki2010--, 胡振震2016,吴凌云2007--}
  \item 档案引用\cite{中国第一历史档案馆2001--}
  \item 未出版物引用\cite{包太雷2013--}
\end{enumerate}

{
%\hyphenation{kurose-gawa}
%\hyphenpenalty=1000
%\tolerance=500
\printbibliography%[heading=subbibintoc,title=【参考文献】]
}
\end{refsection}


\section{测试:学位论文}
\begin{refsection}
\begin{enumerate}
  \item 学位论文引用\cite{CALMS1965--,马欢2011-27-27,吴云芳2003--,张若凌2004--,张志祥1998--}
\end{enumerate}

{
%\hyphenation{kurose-gawa}
%\hyphenpenalty=1000
%\tolerance=500
\printbibliography%[heading=subbibintoc,title=【参考文献】]
}
\end{refsection}


\section{测试:专利}
\begin{refsection}
\begin{enumerate}
  \item 专利引用\cite{KOSEKI2002--,TACHIBANA2002--,河北绿洲生态环境科技有限公司2001--,姜锡洲1989--,刘加林1993--,西安电子科技大学2002--,张凯军2012-04-05--}

  \item 注意:专利文献\{刘加林1993--\}的location定义了中国,但从GB/T 7714-2015中给出的示例看其实并不需要该域,但这里并没有去掉。因为GB/T 7714-2015中给出的著录格式包括了出版项但没有示例,尽管这里的location是专利地域,但也可以作为出版项来考虑。未来国标完善后再做处理。
\end{enumerate}

{
%\hyphenation{kurose-gawa}
%\hyphenpenalty=1000
%\tolerance=500
\printbibliography%[heading=subbibintoc,title=【参考文献】]
}
\end{refsection}

\section{测试:其它类型}
\begin{refsection}
\begin{enumerate}
  \item 其它类型文献引用\cite{gom,gom1,gom2}
\end{enumerate}

{
%\hyphenation{kurose-gawa}
%\hyphenpenalty=1000
%\tolerance=500
\printbibliography%[heading=subbibintoc,title=【参考文献】]
}
\end{refsection}


