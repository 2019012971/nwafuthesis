\section{测试:间距字体颜色控制}
    \begin{refsection}
    {

% 换行的控制
%
% 选项 block=none , space , par , nbpar , ragged
\renewcommand*{\newblockpunct}{\par\nobreak}

% 字体的控制:\textit,sl,emph-楷体,\textbf,sf-黑体,\texttt-仿宋,\textsc,md,up-宋体
%
% 全局字体
\renewcommand{\bibfont}{\zihao{-5}}%\fangsong
% 标题字体
\renewcommand{\bibauthorfont}{\bfseries\color{teal}}%
\renewcommand{\bibtitlefont}{\ttfamily\color{blue}}%
\renewcommand{\bibpubfont}{\itshape\color{violet}}%
% url和doi字体
\def\UrlFont{\ttfamily} %\urlstyle{sf} %\def\UrlFont{\bfseries}

% 间距的控制
\setlength{\bibitemsep}{0ex}
\setlength{\bibnamesep}{0ex}
\setlength{\bibinitsep}{0ex}

% 标点类型的控制(全局字体能控制标点的字体)

%\renewcommand{\thefootnote}{\textcircled{\tiny\arabic{footnote}}}

    文献
\cite{张伯伟2002--}
\cite{白书农1998-146-163}
\cite{杨洪升2013-56-75}
\cite{中华医学会湖北分会1984--}
\cite{雷光春2012--}
\cite{贾东琴2011-45-52}
\cite{汤万金2013-09-30--}
\cite{韩吉人1985-90-99}
\cite{马欢2011-27-27}
\cite{张凯军2012-04-05--}
\cite{国家环境保护局科技标准司1996-2-3}
\cite{中国职工教育研究会1985--}
\cite{丁文祥2000--}
\cite{李强2012-05-03--}

\cite{OBRIEN1994--}
\cite{FOURNEY1971-17-38}
\cite{Park2010-696-715}
\cite{Babu2014--}
\cite{Calkin2011-8-9}
\cite{CALMS1965--}
\cite{KOSEKI2002--}
\cite{standardinfoiso158}
\cite{Dublin2012-06-14--}


%\hyphenation{kurose-gawa}
%\hyphenpenalty=1000
%\tolerance=500
\printbibliography[heading=subbibliography,title=【参考文献】]
}

    \end{refsection}