
% !Mode:: "TeX:UTF-8"
% 用于测试gb7714-2015样式,能否对实现国标要求的脚注文献表
\documentclass{article}
\usepackage{ctex}
\usepackage{xcolor}
\usepackage{toolbox}
\usepackage[colorlinks]{hyperref}
\usepackage{lipsum}
\usepackage[paperwidth=15cm,paperheight=6.5cm,top=10pt,bottom=10pt,left=0.5cm,right=0.5cm]{geometry}
\usepackage{xltxtra,mflogo,texnames}
\usepackage[backend=biber,style=gb7714-2015]{biblatex}

\addbibresource{example.bib}

\begin{document}

\begin{refsection}
示例2:多次引用同一著者的同一文献的脚注序号

……但个人理性选择使得没有人愿意率先违反旧的规范
\footfullcite{Sunstein1996-903-903}。
……事实上,都是民主制度的坚决反对者\footfullcite[20]{Morri2010--}。
……一切后世的思想都是一系列为柏拉图思想所作的脚注\footfullcite{罗杰斯2011-15-16}。
……佛教受到极大的打击\footfullcite[326-329]{Morri2010--}。
……以上谓等威之辨,尊卑之序,由于饮食荣辱。\footfullcite{陈登原2000-29-29}
\end{refsection}


\end{document} 