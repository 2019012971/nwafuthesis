% 使用ctexart文档类(用XeLaTeX编译,直接支持中文)
\documentclass{ctexart}

% 导言区,可以在此引入必要的宏包
\usepackage{texboxie}
% 读取csv数据生成表格的宏包
\usepackage{csvsimple}
\usepackage{pgfplotstable}
\usepackage{datatool}
\usepackage{booktabs}

% 彩色表格
%\usepackage[table]{xcolor}
\usepackage{colortbl}

%%% 表格属性
\colorlet{tableheadcolor}{black!60}
\newcommand\tableheadfont{
  \sffamily\bfseries
  \slshape
  \color{white}
}

\title{通过csv文件生成\LaTeX 表格的几种方法}
\author{耿楠}
\date{\today}

% 生成数据文件
\begin{filecontents*}{db1.csv}
  姓名,性别,年龄
  张三,男,18
  李四,男,45
  马五,女,16
\end{filecontents*}
    
\begin{document} %在document环境中撰写文档
  \maketitle

  \begin{abstract}
    CSV文件是一种用逗号分隔的数据文件,在\LaTeX 中可以采
    用csvsimple、pgfplotstable、datatool、csvtools等宏包直接使用CSV文件
    数据生成\LaTeX 表格。
  \end{abstract}

  \section{准备CSV数据文件}
  CSV数据文件可能用词本、Excel等软件生成,也可以在导言区用如下\LaTeX 代码生成。
  \begin{codeonly}
    \begin{filecontents*}{db1.csv}
      姓名,性别,年龄
      张三,男,18
      李四,男,45
      马五,女,16
    \end{filecontents*}
  \end{codeonly}
  该代码会在当前工作目录下生成``db1.csv''数据文件。

  \section{使用csvsimple宏包}
  \subsection{简单方式}
  可以使用csvautotabular命令直接生成表格:
  \begin{codeonly}
    \csvautotabular{db1.csv}
  \end{codeonly}
  \csvautotabular{db1.csv}

  \subsection{设置表头}
  也可以使用csvreader命令对表头进行设置:
  \begin{codeonly}
    \csvreader[tabular=|l|l|c|,
               table head=\hline & 姓名 & 性别 & 年龄 \\ \hline,
               late after line=\\\hline
              ]%
              {scientists.csv}
              {姓名=\name,性别=\surname,年龄=\age}%
              {\thecsvrow & \surname~\name & \age}%
  \end{codeonly}

  \csvreader[tabular=|c|c|c|c|,
             table head=\hline & 姓名 & 性别 & 年龄\\ \hline,
             late after line=\\\hline
            ]%
            {db1.csv}
            {姓名=\name,性别=\gender,年龄=\age}%
            {\thecsvrow & \name & \gender & \age}%

  关于csvsimple宏包详情,请使用texdoc csvsimple查看其宏包手册。

  \section{使用pgfplotstable宏包}
  \subsection{简单方式}
  可以使用pgfplotstabletypeset命令直接生成表格:
  \begin{codeonly}
    \pgfplotstabletypeset[
      col sep=comma,
      string type,
      columns/name/.style={column name=姓名, column type={|l}},
      columns/gender/.style={column name=性别, column type={|l}},
      columns/age/.style={column name=年龄, column type={|c|}},
      every head row/.style={before row=\hline,after row=\hline},
      every last row/.style={after row=\hline},
      ]{db2.csv}
  \end{codeonly}
  \pgfplotstabletypeset[
	col sep=comma,
	string type,
	columns/name/.style={column name=姓名, column type={|l}},
	columns/gender/.style={column name=性别, column type={|l}},
	columns/age/.style={column name=年龄, column type={|c|}},
	every head row/.style={before row=\hline,after row=\hline},
	every last row/.style={after row=\hline},
	]{db2.csv}

  \subsection{使用multicolumn合并列}
  可以使用pgfplotstabletypeset命令直接生成表格:
  \begin{codeonly}
    \pgfplotstabletypeset[
        col sep=comma,
        string type,
        every head row/.style={%
         before row = {\hline
           \multicolumn{2}{c}{基本信息} & \\
          },
          after row=\hline
        },
        every last row/.style = {after row = \hline},
        columns/name/.style={column name=姓名, column type={l}},
        columns/gender/.style={column name=性别, column type={l}},
        columns/age/.style={column name=年龄, column type={c}},
        ]{db2.csv}
  \end{codeonly}
  \pgfplotstabletypeset[
	col sep=comma,
	string type,
        every head row/.style={%
          before row = {\hline
            \multicolumn{2}{c}{基本信息} & \\
          },
          after row=\hline
        },
        every last row/.style = {after row = \hline},
	columns/name/.style={column name=姓名, column type={l}},
	columns/gender/.style={column name=性别, column type={l}},
	columns/age/.style={column name=年龄, column type={c}},	
	]{db2.csv}
        
  关于pgfplotstable宏包详情,请使用texdoc pgfplotstable查看其宏包手册。                  
  \section{使用datatool宏包}
  \subsection{简单方式}
  可以在使用DTLloaddb载入数据文件后,用DTLdisplaydb命令直接生成表格:

  \begin{codeonly}
    \DTLloaddb[keys={col1,col2,col3}]{mydb}{db1.csv}
    \DTLdisplaydb{mydb}
  \end{codeonly}
  \DTLloaddb[keys={col1,col2,col3}]{mydb}{db1.csv}
  \DTLdisplaydb{mydb}

  \subsection{使用构建复杂表格}
  \DTLloaddb{table}{db2.csv}
  % \begin{table}
  \begin{codeonly}
    \begin{tabular}{llc}
      \toprule      
      姓名 & 性别 & 年龄 \tabularnewline
      \midrule
      \DTLforeach*{table}%
      {\name=name, \gender=gender, \age=age}%
      {\DTLiffirstrow{}{\tabularnewline}%
      \name & \gender & \age}\\
      \bottomrule
   \end{tabular}
  \end{codeonly}
    \begin{tabular}{llc}
      \toprule      
      姓名 & 性别 & 年龄 \tabularnewline
      \midrule
      \DTLforeach*{table}%
      {\name=name, \gender=gender, \age=age}%
      {\DTLiffirstrow{}{\tabularnewline}%
      \name & \gender & \age}\\
      \bottomrule
    \end{tabular}
  %\end{table}

  关于datatool宏包详情,请使用texdoc datatool查看其宏包手册。  
  
  
  
\end{document}

%%% Local Variables:
%%% mode: latex
%%% TeX-master: t
%%% End:
