% 本文件是示例论文的一部分
% 论文的主文件位于上级目录的 `main.tex`

\chapter{快速上手}

\section{欢迎}

欢迎使用 \nwafuthesis,本文档将介绍如何利用 \nwafuthesis 模板进行学位论文写作,
假设读者有 \LaTeX 写作经验,并会使用搜索引擎解决常见问题。

源代码托管于 \url{https://github.com/registor/nwafuthesis},
欢迎来提 issue/PR。

\section{\LaTeX 环境准备}

由于本模板使用了大量宏包,因此对 \LaTeX 环境有不少要求。
推荐使用以下打 \ding{51} 的 \LaTeX 发行版:
\begin{itemize}
\item[\ding{51}]\TeX~Live 请安装以下 collection:langchinese, latexextra, science, pictures, fontsextra;\\
如果觉得安装体积太大的话,可以看 \texttt{.ci/texlive.pkgs} 列出的所需宏包;
\item[\ding{51}]MiK\TeX 可能国内镜像服务器无法联通,如果无法联通,建议隔天再试; \\
因为 MiK\TeX{} 能自动下载安装宏包,推荐 Windows 用户使用。
\item[\ding{53}]CTeX  \qtbr{不推荐},可能会有宏包缺失、版本过旧导致无法编译现象。
\end{itemize}

\section{编译模板和文档}

只有在找不到 \verb|nwafuthesis.cls| 文件的时候,才需要执行本步骤。

进入模板的根目录,运行 \verb|build.bat|(Windows) 或 \verb|build.sh|(其他系统),
它会生成模板 \verb|nwafuthesis.cls| 以及对应的说明文档 \verb|nwafuthesis.pdf|。

\section{使用模板}

论文写作时,请确认\textbf{论文的目录}(\verb|main.tex|所在的目录)下有以下文件:
\begin{itemize}
  \item \verb|nwafuthesis.cls| 文档模板;
  \item \verb|logo/| 文件夹,内含学校的LOGO图标;
\end{itemize}

如果论文目录下没有这些文件的话,请从本模板根目录复制一份。

\section{开始写作}

最方便的开始方法,莫过于修改现有的文稿。因此推荐直接修改本文档,建议按
如下结构组织和管理写作过程中的各个文件:
\begin{itemize}
  \item \verb|main.tex| 主文件,定义了文档包含的内容,建议不要随意更改
    \verb|main.tex|文件名,其它的\verb|*.tex|中会用到该文件名信息;
  \item \verb|setup/packages.tex| 载入需要的宏包,可根据需要进行增
    加或删除;
  \item \verb|setup/format.tex| 全局使用的自定义命令、相关设置等;  
  \item \verb|content/*.tex| 文件夹,按章节拆分的文档内容,分章节以存
    放在这里;
  \item \verb|figs/| 文件夹,插图文件;  
  \item \verb|bib/| 文件夹,内含参考文献数据库,文献数据库是纯文本文
    件,请务必采用\qtbr{UTF8编码}存储;
  \item \verb|ref/| 文件夹,可有可无,内含写作过程中用到的资料、参考文
    献、记录等;  
\end{itemize}

完成部分或所有\verb|*.tex|撰写和修改后,可以在命令行使用 \verb|latexmk -xelatex main|
进行编译输出\verb|main.pdf|文件,可以根据需要对结果\texttt{pdf}文件进行改名。

也可以使用\texttt{TeXstudio}、\texttt{vscode}等图形界面的编辑器的进行
编译输出。

\section{打印论文}

如果论文需要双面打印的话,请务必修改文档类选项,编译双面打印用的 PDF 文件。
具体地说,在主文件的头部,去除 \texttt{openany, oneside},改成 \texttt{openright, blankleft, twoside}。

%%% Local Variables: 
%%% mode: latex
%%% TeX-master: "../main.tex"
%%% End:
