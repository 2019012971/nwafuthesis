% 本文件是示例论文的一部分
% 论文的主文件位于上级目录的 `main.tex`

\chapter{公式与参考文献}

本章节介绍由 \nwafuthesis{} 提供的特有的宏。

\section{定理环境}

\nwafuthesis{} 提供了三个宏 \cs{nwafutheorem(g|chap|chapu)} 以定义不同编号方法的定理环境。
\begin{enumerate}
  \item \cs{nwafutheoremg} 的编号只有一个数字;
  \item \cs{nwafutheoremchap} 的编号由\qtmark{章节.序号}构成,不同定理环境的编号是独立的,
  它们的数字编号会重复,如\qtmark{\autoref{ex:oneplus}}后面可能出现\qtmark{\autoref{non:dora}};
  \item \cs{nwafutheoremchapu} 的编号也是由\qtmark{章节.序号}构成,
  但它们的数字编号是统一的,同一个数字不会重复出现(仅限用\cs{nwafutheoremchapu}声明的定理环境之间)。
  如\qtmark{\autoref{def:distance}}后面\textbf{不会}出现\qtmark{假设~2.1},但可能出现\qtmark{定义~2.2}或\qtmark{\autoref{assume:fail}};
\end{enumerate}

关于用这三个宏定义定理环境的样例请参阅\qtmark{\file{setup/format.tex}}。

由于学校没有规定计数的编号,所以所有的定理环境应该由作者来决定编号方式,
这也意味着所有的定理环境都要由作者来定义。

顺便一提,在同一章里同时出现两种编号方式的定理环境,很可能造成混乱,
所以请合理安排定理环境的编号方式。

\subsection*{样例}

\begin{definition}[欧几里得距离]
\label{def:distance}
点$\mathbf{p}$与点$\mathbf{q}$的\textbf{欧几里得距离},是连接该两点的线段($\overline{\mathbf{pq}}$)的长度。

在笛卡尔坐标系下,如果 $n$维欧几里得空间下的两个点 $\mathbf{p}=(p_1, p_2, \dots, p_n)$ 与点
$\mathbf{q} = (q_1, q_2, q_3, \dots, q_n)$,那么点$\mathbf{p}$与点$\mathbf{q}$的距离,
或者点$\mathbf{q}$与点$\mathbf{p}$的距离,由以下公式定义:
\begin{align}
\label{equ:1}
d(\mathbf{p},\mathbf{q}) = d(\mathbf{q},\mathbf{p}) & = \sqrt{(q_1-p_1)^2 + (q_2-p_2)^2 + \cdots + (q_n-p_n)^2} \\
\label{equ:2}
& = \sqrt{\sum_{i=1}^n (q_i-p_i)^2}
\end{align}
\end{definition}

\begin{proof}
由\cs{nwafutheorem(g|chap|chapu)}定义的定理环境支持 \cs{autoref},
比如在\autoref{def:distance}中,\autoref{equ:2}是\autoref{equ:1}的简写。

但是 \cs{autoref} 只能在 \cs{ref} 加上前缀,无法加上后缀。
所以上一句话的后半部分,更推荐手工来写标注 “(\ref{equ:2}) 是 (\ref{equ:1}) 的简写”。

定理环境里面可以换行,不过证明与其他定理环境稍有不同,它是单独定义实现的,
因此末尾会有一个 QED 符号。
\end{proof}

\begin{assumption}
\label{assume:fail}
假设本身就不成立
\end{assumption}

\begin{lines}
\label{s1}
例句1
\end{lines}

与图表一样,公式、定理等也需要采用专用的命令或环境进行排版以实现
编号、交叉引用等\qtb{自动化}处理,\qtbr{万万不可}手动编号、引用!

\section{参考文献}
\label{sec:bib}
参考文献的引用采用\qtmark{著者-出版年}制,如:

\subsection{引用方式}
\subsubsection{著者作为引用主语}

文中提及著者,在被引用的著者姓名或外国著者姓氏之后用
圆括号标注文献出版年,可使用\cs{yearcite}、\cs{textcite}
命令或手动模式引用文献,如:

\begin{center}
  \begin{minipage}[h]{0.9\linewidth}
    \begin{texdemov}%[0.5]
赵耀东\yearcite{赵耀东1998--}认为...;
\textcite{赵耀东1998--}认为...;
赵耀东(\cite*{赵耀东1998--})认为...;
赵耀东(\citeyear{赵耀东1998--})认为...;
    \end{texdemov}
  \end{minipage}
\end{center}

\emph{注意}:手动模式使用\cs{cite*}或\cs{citeyear}命令时,需要在两端加上小括
  号,\qtbr{推荐使用}\cs{textcite}命令。

\subsubsection{提及内容未提及著者}

文中只提及所引用的资料内容而未提及著者,则在引文叙述
文字之后用圆括号标注著者姓名或外国著者姓氏和出版年份,在著者
和年份之间空一格,此时可以使用\cs{cite}命令引用文
献,如:

\begin{center}
  \begin{minipage}[h]{0.9\linewidth}    
    \begin{texdemov}%[0.5]      、
孟德尔发现了一个很重要的现象,即红、白花豌豆杂交后的所结种子
第二年长出的植株的红白花比例为3:1\cite{fzx1962}。%
    \end{texdemov}
  \end{minipage}
\end{center}

\subsubsection{同一著者多篇文献}

引用同一著者不同年份出版的多篇文献时,后者只注出版年;
引用同一著者在同一年份出版的多篇文献时,无论正文还是文末,年
份之后用英文小写字母 a、b、c 等加以区别。按年份递增顺序排列,
不同文献之间用逗号隔开。此时可以使用\cs{cite}命令引用文
献,如:

\begin{center}
  \begin{minipage}[h]{0.9\linewidth}    
    \begin{texdemov}%[0.5]      、
      UML基础和Rose建模教程中给出了大量案例及案例分析\cite{蔡敏2006a--,蔡敏2006b--}。%
    \end{texdemov}
  \end{minipage}
\end{center}

\subsubsection{两著者文献}

引用两个著者的文献时,两个著者之间加\qtmark{和}(中文)或
\qtmark{and}(英文)。此时可以使用\cs{cite}命令引用文
献,如:

\begin{center}
  \begin{minipage}[h]{0.9\linewidth}    
    \begin{texdemov}%[0.5]      、
利用基于Matlab的计算机仿真\cite{郭文彬2006--},研究了UWB和窄带通讯中的信号共存特性\cite{Chiani2009-231-254}。%
    \end{texdemov}
  \end{minipage}
\end{center}

\subsubsection{三个以上著者文献}

引用三个以上著者时,只标注第一著者姓名,其后加\qtmark{等}(中
文)或\qtmark{et al.}(英文)。此时可以使用\cs{cite}命令引用文
献,如:

\begin{center}
  \begin{minipage}[h]{0.9\linewidth}    
    \begin{texdemov}%[0.5]      、
      UML基础和Rose建模教程中详细说明了其基本方法和技巧\cite{蔡敏2006--}。你不好好学点\LaTeX{}基本命令还真不行\cite{r9}。%
    \end{texdemov}
  \end{minipage}
\end{center}

\subsubsection{同一处引用多篇文献}

同一处引用多篇文献时,按著者字母顺序排列,不同著者文
献之间用分号隔开。此时可以使用\cs{cite}命令引用文
献,注意用\qtbr{逗号}分开\texttt{citeKey}就好,如:

\begin{center}
  \begin{minipage}[h]{0.9\linewidth}    
    \begin{texdemov}%[0.5]      、
      同时引用多个文献\cite{r2,r3,r4,r6}。%
    \end{texdemov}
  \end{minipage}
\end{center}
  
\subsubsection{多次引用同一著者的同一文献}

多次引用同一著者的同一文献,在正文中标注著者与出版年,
并在\qtmark{()}内以以冒号形式标注引文页码。此时可以使用\cs{parencite}命令引用文
献,注意用可选参数指定引用页码,如:

\begin{center}
  \begin{minipage}[h]{0.9\linewidth}    
    \begin{texdemov}%[0.5]      、
      在文献\parencite[20-22]{程根伟1999-32-36}说了一, 在文献\parencite[55-60]{程根伟1999-32-36}说了二。%
    \end{texdemov}
  \end{minipage}
\end{center}

\subsection{输出参考文献列表}

参考文献列表的输出只需在需要输出文献的位置,使用命令\cs{printbibliography}进行输出即可。

\subsection{参考文献数据文件准备}

\LaTeX 文档中生成参考文献一般都需要准备一个参考文献数据源文件即
\qtmark{*.bib}文件。这一文件内保存有各条参考文献的信息,具体可以参考
biblatex宏包手册和biblatex-gb7714-2015样式包手册\cite{胡振震2019}中关
于域信息录入的说明。

参考文献源文件本质上只是一个文本文件,只是其内容需要遵守BibTeX格式,参
考文献源文件可以有多种生成方式,具体可参考\LaTeX{}文档中文参考文献的
biblatex 解决方案\parencite[2.2节]{胡振震2016}。

\note{学校的参考文献格式并不完全符合\texttt{GB7714-2015}参考文献著录标准,强烈建议学校参考文
  献执行\texttt{GB7714-2015}参考文献著录标准!}

本模板采用由胡振震维护的
\qtmark{符合 GB/T 7714-2015 标准的 biblatex 参考文献样式}实现参考文献
的编排\cite{胡振震2019},其Github链接为
\url{https://github.com/hushidong/biblatex-gb7714-2015}。
大家也可以通过\TeX~Live的 \verb|texdoc gb7714-2015|命令查看其使用说明。

关于著者-出版年样式命令的详细说明可参见胡振震\qtmark{符合 GB/T
  7714-2015 标准的 biblatex 参考文献样式}说明中的中的相关内容
\parencite[2.2、2.3节]{胡振震2016}。

\qtbr{切记:}与图表、公式、定理等一样,请使用专用命令引用并输出参考文
献,以实现参考文献的\qtb{自动化}处理,\qtbr{万万不可}手动编写参考文献!


%%% Local Variables: 
%%% mode: latex
%%% TeX-master: "../main.tex"
%%% End:
